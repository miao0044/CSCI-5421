\documentclass[12pt]{amsart}
\usepackage[totalwidth=480pt, totalheight=680pt]{geometry}
\usepackage{amsmath, amssymb, amsthm, enumerate}
\setlength\parindent{0pt}
\usepackage[dvips]{graphicx}
\usepackage{graphicx}
\usepackage{float}
\usepackage{amsmath,indentfirst,amsfonts,dsfont,amssymb,amsthm,graphicx,enumerate,mathrsfs,color,xspace,epstopdf}
\usepackage{algorithm, algorithmic, subfigure}
\numberwithin{equation}{section}



\newcommand{\bb}{\begin{equation}}
\newcommand{\ee}{\end{equation}}
\newcommand{\Z}{\mathbb{Z}}

\renewcommand{\(}{\left(}
\renewcommand{\)}{\right)}

\renewcommand{\[}{\left[}
\renewcommand{\]}{\right]}


%\renewcommand{\mod}[1]{\quad (\textrm{mod}\ #1)}

\font\srm=cmr8 \font\ssi=cmti8
\font\bigbf=cmbx12 \font\ssr=cmss8 \font\sst=cmtt8 \font\tt=cmtt10
\font\ssb=cmbx8 \font\srm=cmr8 \font\ssi=cmti8 \font\sar=cmss12
\font\srs=cmr6



\renewcommand{\theequation}{\thesection.\arabic{equation}}

\begin{document}
\pagestyle{empty}

\noindent \textbf{31.5-1} According to Chinese Remainders theorem, the solution is to modulo $55$. We find $M_1 = 5^{-1} \mod 11 = 9$ and $M_2 = 11^{-1} \mod 5 = 1$, and therefore $c_1=5M_1=45$ and $c_2=11M_2=11$. Therefore, one solution is given by $x = (5c_1 + 4c_2) \mod 55= (5\cdot 45+4\cdot 11)\mod 55 =49$, and the rest are of the form $55k+49$ for $k\in \Z$.

\bigskip

\noindent \textbf{31.5-2} According to Chinese Remainders theorem, the solution is to modulo $9\cdot 8\cdot 7 = 504$. We find, $m_1=8\cdot 7=56$, $m_2=9\cdot 7=63$ and $m_3=9\cdot 8=72$. The inverses are $M_1 = 56^{-1} \mod 9= 2^{-1} \mod 9 = 5$, $M_2=63^{-1} \mod 8 = 7^{-1} \mod 8=7$ and $M_3=72^{-1} \mod 7 = 2^{-1} \mod 7=4$. One solution is then given by
$$
x=(1\cdot 56 \cdot 5 + 2 \cdot 63 \cdot 7 + 3 \cdot 72 \cdot 4) \mod 504 = 2026 \mod 504 = 10
$$
and the rest are $504k+10$ for $k\in \Z$.

\bigskip

\noindent \textbf{31.5-3} Denote by $x=a^{-1} \mod n$ and $x_i = x \mod n_i$. Since $ax \mod n = 1$ that means $ax \mod n_i=1$ for every $n_i$. But since $a=c_1a_1+\ldots+c_ka_k$ and $n_i \mid c_j$ for every $j\neq i$, it further implies that
$$
1=ax \mod n_i = (c_1a_1+\ldots+c_ka_k)x\mod n_i = c_ia_ix \mod n_i = a_ix \mod n_i
$$
since $c_i\mod n_i = (m_i(m_i^{-1} \mod n_i)) \mod n_i = 1$. Now if $x_i = x\mod n_i$, previous equation implies $a_ix_i \mod n_i=1$.

\medskip

Otherwise, let $x$ be the number such that $x_i:=x\mod n_i = a_i^{-1} \mod n_i$. Then $x=p_in_i+x_i$, for some $p_i$ and
$$
ax = \sum_{l=1}^k a_lc_l(p_ln_l+x_l)= \sum_{l=1}^k a_lm_lM_l(p_ln_l+x_l) \equiv_{n_i} a_im_iM_ix_i=1
$$
where $M_i = m_i^{-1} \mod n_i$ and since $n_i \mid m_l$ for $i\neq l$. Then, $n_i \mid ax-1$ and since $n_i$ are relative prime, it implies $n\mid ax-1$ and $ax\mod n=1$.

\bigskip

\noindent \textbf{31.6-1} Smallest primitive root is $2$, since
$$
\begin{array}{ccccccccccc}
i&  1 & 2 & 3 & 4 & 5 & 6 & 7 & 8 & 9 & 10 \\
2^i& 2 & 4 & 8 & 16 & 32 & 64 & 128 & 256 & 512 & 1024 \\
2^i \mod 11 & 2 & 4 & 8 & 5 & 10 & 9 & 7 & 3 & 6 & 1
\end{array}
$$
and therefore
$$
\begin{array}{ccccccccccc}
a & 2 & 4 & 8 & 5 & 10 & 9 & 7 & 3 & 6 & 1 \\
\mathrm{ind}_{11,2}(a)&  1 & 2 & 3 & 4 & 5 & 6 & 7 & 8 & 9 & 10
\end{array}
$$
Now since $\mathrm{ord}_{11}(a) = 10/\gcd(\mathrm{ind}_{11,2}(a),10)$ ($\phi(11)=10$) we find:
$$
\begin{array}{ccccccccccc}
a&  1 & 2 & 3 & 4 & 5 & 6 & 7 & 8 & 9 & 10 \\
\mathrm{ord}_{11}(a)& 1 & 10 & 5 & 5 & 5 & 10 & 10 & 10 & 5 & 2
\end{array}
$$

\newpage


\noindent \textbf{31.6-2} Value of $p$ in $i$-th iteration is $a^{2^i}$.

\begin{algorithm}[H]
\caption{ModExp($a,b,n$)}
\begin{algorithmic}[1]
\REQUIRE $(b_k,b_{k-1},\ldots,b_0)$, the binary representation of $b$.
\STATE $p:=a$, $x:=1$
\FOR{$i=0$ to $k$}
\IF{$b_i=1$}
\STATE $x:=(p\cdot x) \mod n$
\ENDIF
\STATE $p:=p^2 \mod n$
\ENDFOR
\STATE \textbf{return} x
\end{algorithmic}
\end{algorithm}

\bigskip

\noindent \textbf{31.6-3} Euler theorem implies $a^{\phi(n)}\equiv_n1$. Denote $x=a^{\phi(n)-1}$. Then $ax\equiv_n1$ implying $x=a^{\phi(n)-1}=a^{-1} \mod n$. Therefore, one can compute $a^{-1}:=\textsc{Modular-Exponentiation}(a, \phi(n)-1, n)$.

\bigskip

\noindent \textbf{31.7-3} By definition of $P_A(M)=M^e \mod n$, we have that
$$
P_A(M_1)P_A(M_2)=M_1^eM_2^e \mod n = (M_1M_2)^e \mod n = P_A(M_1M_2)
$$

Let $C$ be the ciphertext and $Dec$ is the set of ciphers, adversary can decrypt. He should follow the procedure:
\begin{enumerate}
  \item $i:=0$
  \item If $C \in Dec$, decrypt it. Let $M$ be message.
  \item Choose random $x_i$ and compute $C:=C\cdot P_A(x_i)$
  \item $i:=i+1$ and go to step 1
  \item Return $Mx_{i-1}^{-1}\cdots x_0^{-1} \mod n$
\end{enumerate}
Inverses modulo $n$ can be obtained by Modular-Exponentiation, since $x^{-1} \mod n = x^{n-2} \mod n$. Since $Dec$ contains about $1/100$-th part of all ciphers, it would need about 100 iterations of the previous algorithm to terminate. So adversary can do the last step in reasonable time.

\bigskip

\noindent \textbf{31.8-3} Let $p$ be arbitrary prime and $\alpha,m$ integers. Denote by $p^\alpha \parallel m$ if $p^\alpha \mid m$ but $p^{\alpha+1} \nmid m$. Since $x^2 \equiv_n1$ then $n\mid x^2-1$ and $\gcd(x^2-1,n)=n$. On the other hand, $p^\alpha \parallel x-1$, $p^\beta \parallel x+1$ and $p^\gamma \parallel n$, then $p^{\min\{\alpha,\gamma\}}\parallel \gcd(x-1,n)$ and $p^{\min\{\beta,\gamma\}}\parallel \gcd(x+1,n)$. On the other hand, since $x^2-1=(x-1)(x+1)$, we have $p^{\alpha+\beta}\parallel x^2-1$ and hence $p^{\min\{\alpha+\beta,\gamma\}}\parallel \gcd(x^2-1,n)$. Since $\min\{\alpha,\gamma\}+\min\{\beta,\gamma\}\geq \min\{\alpha+\beta,\gamma\}$, and previous equations holds for arbitrary prime $p$, we conclude that 
$$
n=\gcd(x^2-1,n) \mid \gcd(x-1,n)\gcd(x+1,n) 
$$
Now since $1<x<n-1$, neither $\gcd(x-1,n)$ nor $\gcd(x+1,n)$ cannot be $n$. Therefore, if either of these factors is $1$, previous equation implies that the other is $n$, which is contradiction. Hence, they are both in $(1,n)$ and hence non-trivial. 



\end{document} 