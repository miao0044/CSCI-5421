\documentclass[12pt]{amsart}
\usepackage[totalwidth=480pt, totalheight=680pt]{geometry}
\usepackage{amsmath, amssymb, amsthm, enumerate}
\setlength\parindent{0pt}
\usepackage[dvips]{graphicx}
\usepackage{graphicx}
\usepackage{float}
\usepackage{amsmath,indentfirst,amsfonts,dsfont,amssymb,amsthm,graphicx,enumerate,mathrsfs,color,xspace,epstopdf}
\usepackage{algorithm, algorithmic, subfigure}
\numberwithin{equation}{section}



\newcommand{\bb}{\begin{equation}}
\newcommand{\ee}{\end{equation}}

\renewcommand{\(}{\left(}
\renewcommand{\)}{\right)}

\renewcommand{\[}{\left[}
\renewcommand{\]}{\right]}


%\renewcommand{\mod}[1]{\quad (\textrm{mod}\ #1)}

\font\srm=cmr8 \font\ssi=cmti8
\font\bigbf=cmbx12 \font\ssr=cmss8 \font\sst=cmtt8 \font\tt=cmtt10
\font\ssb=cmbx8 \font\srm=cmr8 \font\ssi=cmti8 \font\sar=cmss12
\font\srs=cmr6



\renewcommand{\theequation}{\thesection.\arabic{equation}}

\begin{document}
\pagestyle{empty}

\noindent \textbf{30.1-2} Since $A(x) = (x-x_0) q(x) +r$ and
$$
A(x) = \sum_{i=0}^n a_i x^i, \qquad q(x) = \sum_{j=0}^{n-1} q_jx^j
$$
then
$$
\aligned
(x-x_0) q(x) +r &= (x-x_0) \sum_{i=0}^{n-1} q_jx^j + r\\
&= q_{n-1}x^n + (q_{n-2} + x_0 q_{n-1})x^{n-1} + (q_{n-3} + x_0 q_{n-2})x^{n-2} + \ldots + (q_0 + x_0 q_1)x + q_0x_0 + r
\endaligned
$$
Equating coefficients on both sides of $A(x) = (x-x_0) q(x) +r$, one obtains
$$
\aligned
q_{n-1} &= a_n\\
q_{n-2} &= a_{n-1} - x_0 q_{n-1}\\
&\ \vdots\\
q_{n-k} &= a_{n-k+1} - x_0 q_{n-k+1}\\
&\ \vdots\\
q_0 &= a_1 - x_0 q_1\\
r &= a_0 - x_0 q_0
\endaligned
$$
Therefore, setting $q_{n-1} := a_n$ and applying formulas $q_{n-k} := a_{n-k+1} - x_0 q_{n-k+1}$ for $k=1, 2, \ldots, n$ gives an $\Theta(n)$ algorithm for computing $q(x)$ and $r$ by $r := a_0 - x_0 q_0$

\bigskip

\noindent \textbf{30.1-3} Assume that $(x_i, a_i)$ ($i=0,1,\ldots,n-1$) is given point-value representation of $A(x)$, i.e. $a_i=A(x_i)$. Since $A^{\textrm{rev}}(x) = x^{n-1} A(x^{-1})$ then $A^{\textrm{rev}}(x_i^{-1}) = x_i^{-(n-1)} A(x_i) = x_i^{-(n-1)} a_i$ for every $i=0,1,\ldots,n-1$. In other words $(x_i, x_i^{-(n-1)}a_i)$  ($i=0,1,\ldots,n-1$) is required point-value representation of $A^{\textrm{rev}}(x)$.

\bigskip

\noindent \textbf{30.1-4} Assume that $(x_i, p_i)$ ($i=0,1,\ldots,k$) are given point-value pairs where $k<n-1$. Choose arbitrary distinct $x_{k+1}, x_{k+2}, \ldots, x_{n-1} \in \mathbb{R} \setminus \{ x_0,x_1,\ldots,x_k \}$ and arbitrary values $p_{k+1}, p_{k+2},\ldots,p_{n-1}$. Then, according to Theorem 30.1 exists a unique polynomial $p(x)$ of degree-bound $n$ satisfying $p(x_i)=p_i$ for all $i=0,1,\ldots,n-1$. Now choosing different $p'_{k+1}, p'_{k+2},\ldots,p'_{n-1}$, in the same way, one gets a new polynomial $p'(x)$ such that $p(x_i)=p_i$ for all $i=0,1,\ldots,k$ and $p(x_i)=p'_i$ for all $i=k+1,k+2,\ldots,n-1$. Now $p'(x_i)=p_i' \neq p_i=p(x_i)$ ($i=k+1,k+2,\ldots,n-1$) implies that $p$ and $p'$ are distinct polynomials passing through point-value pairs $(x_i, p_i)$ ($i=0,1,\ldots,k$).

\bigskip

\noindent \textbf{30.1-7} Let us introduce polynomials
$$
P_A(x) = \sum_{i \in A} x^i, \qquad P_B(x) = \sum_{j \in B} x^j.
$$
Then
$$
Q(x) = P_A(x)P_B(x) = \[ \sum_{i \in A} x^i \] \[ \sum_{j \in B} x^j \] = \sum_{\begin{subarray}{c} i \in A\\ j \in B \end{subarray}} x^{i+j}
 = \sum_{k=0}^{20n} c_k x^k
$$
where $c_k$ is the number of appearances of the term $x^k$ in the previous sum, for $k=0,1,\ldots,20n$. Since each $k$ is obtained as the sum $k=i+j$, $i\in A$ and $j\in B$, we conclude that $C = \{k \mid c_k>0,\ k=0,1,\ldots,2n\}$ and that $c_k$ is required number of times.

\medskip

Coefficients $c_k$ of $Q(x)$ can be computed in $\mathcal{O}(n \lg n)$ time using FFT polynomial multiplication.

\bigskip

\noindent \textbf{30.2-1} $\omega_n^{n/2} = \[e^{2\pi i/n}\]^{n/2} = e^{2\pi i/n \cdot n/2} = e^{\pi i} = -1 = e^{2\pi i/2} = \omega_2$

\bigskip

\noindent \textbf{30.2-5} Consider the polynomial $A(x)=\sum_{k=0}^{n-1}a_kx^k$ with degree $n-1$ and write it in the following form:
$$
A(x) = A^0(x^3) + xA^1(x^3) + x^2A^2(x^3)
$$
where
$$
A^j(x) = \sum_{k=0}^{n/3-1} a_{3k+j}x^k, \qquad j=0,1,2.
$$
The goal is to evaluate $A(x)$ at the points $\omega_n^0, \omega_n^1, \ldots,\omega_n^{n-1}$. Since
$$
A(\omega_n^i) = A^0(\omega_n^{3i}) + \omega_n^iA^1(\omega_n^{3i}) + \omega_n^{2i}A^2(\omega_n^{3i})
$$
and $\omega_n^{3i}=\omega_{n/3}^i$, it is enough to evaluate $A^j(x)$ in $\omega_{n/3}^i$ for $i=0,1,\ldots,n/3-1$ and $j=0,1,2$ and then one can compute required values $A(\omega_n^i)$ ($i=0,1,\ldots,n-1$) in $\mathcal{O}(n)$ time.

The whole procedure is given by the following pseudocode:

\begin{algorithm}[H]
\caption{RecFFT3($\mathbf{a}$)}
\begin{algorithmic}[1]
\REQUIRE $\mathbf{a} = (a_0,a_1,\ldots,a_{n-1})$, the coefficients of the polynomial $A(x)$.
\IF{$n=1$}
\STATE \textbf{return} $a_0$
\ENDIF
\STATE $\omega_n := e^{2\pi i/n}$, \quad $\omega := 1$
\STATE $\mathbf{a}^0 := (a_0, a_3, \ldots, a_{n-3})$, \quad $\mathbf{y}^0 := \mathrm{RecFFT}(\mathbf{a}^0)$
\STATE $\mathbf{a}^1 := (a_1, a_4, \ldots, a_{n-2})$, \quad $\mathbf{y}^1 := \mathrm{RecFFT}(\mathbf{a}^1)$
\STATE $\mathbf{a}^2 := (a_2, a_5, \ldots, a_{n-1})$, \quad $\mathbf{y}^2 := \mathrm{RecFFT}(\mathbf{a}^2)$
\FOR{$k=0$ to $n-1$}
\STATE    $k' := k \textrm{ mod } n/3 $  \qquad (corresponds to $\omega_n^{3k} = \omega_{n/3}^k = \omega_{n/3}^{k \mod n/3}$)
\STATE    $y_k := y^0_{k'} + \omega \cdot y^1_{k'} + \omega^2 \cdot y^2_{k'}$
\STATE    $\omega := \omega \omega_n$
\ENDFOR
\STATE \textbf{return} $\mathbf{y} = (y_0, y_1, \ldots, y_{n-1})$
\end{algorithmic}
\end{algorithm}

Denote by $T(n)$ the time complexity of the previous algorithm. Since there are 3 recursive calls and additional $\mathcal{O}(n)$ operations, the recurrence for $T(n)$ is given by:
$$
T(n) = 3T(n/3) + \mathcal{O}(n)
$$
which solution is given by $\mathcal{O}(n \log_3 n)$.


\bigskip

\noindent \textbf{30.2-7} The idea is to divide the roots $(z_0,z_1,\ldots,z_{n-1})$ in two halves, compute the polynomial for both halves recursively, and then multiply obtained polynomials using FFT based multiplication.

\begin{algorithm}[H]
\caption{PolyFromRoots($\mathbf{z}$, $x$)}
\begin{algorithmic}[1]
\REQUIRE $\mathbf{z} = (z_0,z_1,\ldots,z_{n-1})$ and symbolic variable $x$.
\STATE $k := \lfloor n/2 \rfloor$
\STATE $\mathbf{z'} := (z_0,z_1,\ldots,z_{k-1})$
\STATE $\mathbf{z''} := (z_k,z_{k+1},\ldots,z_{n-1})$
\STATE $P'(x) := \mathrm{PolyFromRoots}(\mathbf{z}', x)$
\STATE $P''(x) := \mathrm{PolyFromRoots}(\mathbf{z}'', x)$
\STATE {\bf return} $\textrm{FFT-MUL}(P'(x), P''(x))$
\end{algorithmic}
\end{algorithm}

Note that the time complexity $T(n)$ of the previous algorithm satisfies
$$
T(n) = 2T(n/2) + \mathcal{O}(n \lg n)
$$
where $n \lg n $ term comes from FFT based multiplication. For the sake of simplicity, assume that $n=2^k$. Then
$$
\aligned
T(2^k) &= 2T(2^{k-1}) + C \cdot k \cdot 2^k \\
&= 4T(2^{k-2}) + C \cdot (k-1)2^k + k \cdot 2^k\\
& \ \vdots \\
&= 2^kT(1) + C \cdot (1+2+\ldots+k)\cdot2^k \\
&= 2^kT(1) + C \cdot 2^{k-1}k(k+1) \\
\endaligned
$$
implying $T(2^k) = \mathcal{O}(2^k k^2) = \mathcal{O}(n \lg^2 n)$.




\end{document} 